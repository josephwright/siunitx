\iffalse meta-comment

File: siunitx.tex Copyright (C) 2014-2015 Joseph Wright

It may be distributed and/or modified under the conditions of the
LaTeX Project Public License (LPPL), either version 1.3c of this
license or (at your option) any later version.  The latest version
of this license is in the file

   http://www.latex-project.org/lppl.txt

This file is part of the "siunitx bundle" (The Work in LPPL)
and all files in that bundle must be distributed together.

The released version of this bundle is available from CTAN.

-----------------------------------------------------------------------

The development version of the bundle can be found at

   http://github.com/josephwright/siunitx

for those people who are interested.

-----------------------------------------------------------------------

\fi

\documentclass{l3doc}

% The next line is needed so that \GetFileInfo will be able to pick up
% version data (quite apart from making the demos work).
\usepackage{siunitx}

% Commands for this document
\usepackage{xparse}
\NewDocumentCommand\foreign{m}{\emph{#1}}

\begin{document}

\GetFileInfo{siunitx.sty}

\title{%
  \pkg{siunitx} -- A comprehensive (SI) units package%
  \thanks{This file describes \fileversion,
    last revised \filedate.}%
}

\author{%
  Joseph Wright%
  \thanks{%
    E-mail:
    \href{mailto:joseph.wright@morningstar2.co.uk}
      {joseph.wright@morningstar2.co.uk}%
  }%
}

\date{Released \filedate}

\maketitle
 
\tableofcontents

\begin{documentation}

\section{Introduction}

The correct application of units of measurement is very important in technical
applications. For this reason, carefully-crafted definitions of a coherent
units system have been laid down by the \foreign{Conf\'erence G\'en\'erale des
Poids et Mesures} (CGPM): this has resulted in the \foreign{Syst\`eme
International d'Unit\'es}~(SI). At the same time, typographic conventions for
correctly displaying both numbers and units exist to ensure that no loss of
meaning occurs in printed matter.

The \pkg{siunitx} package aims to provide a unified method for \LaTeX{} users
to typeset numbers and units correctly and easily. The design philosophy of
\pkg{siunitx} is to follow the agreed rules by default, but to allow variation
through option settings. In this way, users can use \pkg{siunitx} to follow the
requirements of publishers, co-authors, universities, \foreign{etc}.\ without
needing to alter the input at all.

\section{Thanks}

Many users have provided feedback, bug reports and ideas for new features for
\pkg{siunitx}: thanks to all of them. Particular thanks to Stefan Pinnow, who
has taken the lead role as beta tester for \pkg{siunitx}, finding incorrect
output, bad documentation and the odd spelling mistake in the documentation.
Thanks also to Enrico Gregorio for encouraging me to complete a fully
\pkg{expl3}-compliant version of the package. Thanks also to Danie Els and
Marcel Heldoorn for the \pkg{SIstyle} and \pkg{SIunits} packages, respectively,
which provided the starting point for the development of \pkg{siunitx}.

\end{documentation}

\PrintIndex

\end{document}